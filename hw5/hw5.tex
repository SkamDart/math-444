% Cameron Dart - Math 444 - Homework 5 - Summer 2017
% Section 3.5: 6, 7
% Section 3.7: 3b, 8, 11
\documentclass[12pt]{article}
\usepackage[margin=1in]{geometry} 
\usepackage{amsmath,amsthm,amssymb,amsfonts}
\usepackage{enumitem} 
\usepackage{cancel}

\newcommand{\N}{\mathbb{N}}
\newcommand{\Z}{\mathbb{Z}}
\newcommand{\R}{\mathbb{R}}
\newcommand{\nifty}{\lim_{n \rightarrow \infty}}
\newenvironment{claim}[2][Claim]{\begin{trivlist}
		\item[\hskip \labelsep {\bfseries #1}\hskip \labelsep {\bfseries #2}]}{\end{trivlist}}

\begin{document}
	\title{Math 444 - Homework 4}
	\author{Cameron Dart}
	\maketitle

\begin{claim}{3.5.6}
	There exists a non cauchy sequence that satisfies $\nifty |x_{n+1} - x_n| < 0$.
\end{claim}
\begin{proof}
	Consider the sequence $x_n$ defined as follows,
	\[ x_n = \sum_{n = 0}^{\infty} \frac{1}{n} \]
	Clearly, \[ \nifty \frac{1}{n} \rightarrow 0 \text{ and } \nifty \frac{1}{n + 1} \rightarrow 0 \text{ so}  \nifty | x_{n + 1} - x_n | = 0 \]
	But $x_n$ diverges so there must exist a non cauchy sequence that satisfies  $\nifty |x_{n+1} - x_n| < 0$.
\end{proof}

\begin{claim}{3.5.7}
	If $x_n$ is a cauchy sequence so that $x_n \in \Z$ , then $x_n$ is constant
\end{claim}
\begin{proof}
	Assume to contradiction that $x_n$ is a non constant cauchy sequence contained in $\Z$. Since $x_n$ is cauchy by definition for all $\epsilon >0$ there exists $k = k(\epsilon)$ so that if $m,n \geq k$, then
	$|x_m - x_n| < \epsilon$.  Since $x_n$ is contained in $\Z$, $|x_m - x_n| >= 1$ for all $x_m,x_n$ where $x_m \neq x_n$. This contradicts our previous statement that $|x_m - x_n| < \epsilon$ for all $\epsilon > 0$. Thus, it must be true that $x_m = x_n$ and $x_n$ is constant.
\end{proof}

\begin{claim}{3.7.3b}
	\[ \sum_{n = 0}^{\infty} = \frac{1}{(\alpha + n)(\alpha + n + 1)} = \frac{1}{\alpha} > 0, \text{if } \alpha > 0. \]
\end{claim}
\begin{proof}
	\begin{align*}
		\sum_{n = 0}^{\infty} = \frac{1}{(\alpha + n)(\alpha + n + 1)} &= \frac{1}{(\alpha+ 1)} - \frac{1}{(\alpha + n + 1)}\\
		&= \frac{1}{\alpha} - \cancel{\frac{1}{\alpha + 1}} + \cancel{\frac{1}{\alpha + 1}} - \cancel{\frac{1}{\alpha + 2}} + ... + \cancel{\frac{1}{\alpha  + n}} - \frac{1}{\alpha + n + 1}\\
		& = \frac{1}{\alpha} - \frac{1}{\alpha + n + 1}\\
		& \nifty  \frac{1}{\alpha} - \frac{1}{\alpha + n + 1} = \frac{1}{\alpha} \,\,\,\,\, \forall \alpha > 0
	\end{align*}
\end{proof}

\begin{claim}{3.7.8}
	
\end{claim}
\begin{proof}
\end{proof}

\begin{claim}{3.7.11}
	If $\Sigma a_n$ is convergent, then $\Sigma a_n^2$ must converge.
\end{claim}
\begin{proof}
	Suppose $\Sigma a_n$ converges to some $L \in R$. Note,
	\[  \sum_{n = 0}^{\infty} a_n^2 \leq \left (\sum_{n = 0}^{\infty} a_n \right )^2 = L^2\]
	Let $a_n' = \left (\sum_{n = 0}^{\infty} a_n \right )^2 $ and apply the ratio test to $a_n$ and $a_n'$
	\[ \nifty \frac{a_n'}{a_n}  = \frac{L^2}{L} = L\]
	Thus, the ratio test says, $a_n'$ must converge since $a_n$ converges. Lastly, $a_n^2$ converges when we apply the comparison test to it with $a_n'$. 
\end{proof}

\end{document}